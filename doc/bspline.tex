\documentclass{article}

\usepackage{graphics}
\usepackage{color}

\title{B-splines in Roboptim}
\author{Florent Lamiraux, CNRS-LAAS}

\begin{document}
\maketitle
\section{Definition}
\begin{figure}[h]
\centerline{
  \input{pictures/basis-functions.pdf_t}
}
\caption{A cubic B spline is defined over interval $[t_3,t_m]$ by $m$ control points.}
\end{figure}
B-splines in \textsf{roboptim} are defined as follows. Given
\begin{itemize}
      \item a number $m\geq 4$ of control points,
      \item regularly spaced time points: $t_0 < t_1 < \cdots < t_{m}$, $\forall i\in\{0,...,m-1\}$, $t_{i+1}-t_i = \Delta t$
      \item $m$ control points $ \textbf{P}_0,\cdots, \textbf{P}_{m-1}$ in $\textbf{R}^n$ ,
\end{itemize}

      the cubic B-spline of control points $\textbf{P}_0,\cdots, \textbf{P}_{m-1}$ is defined over $[t_3,t_{m}]$ by
      \begin{equation}\label{eq:def-spline}
        B(t) = \sum_{i=0}^{m-1} \textbf{P}_i b_{i,3}(t)
      \end{equation}
      where basis functions $b_{i,3}$ are defined by:
      \begin{eqnarray*}
      b_{i,3}(t)=& \frac{(t-t_i)^3}{6\ \Delta t^3} & \mbox{ if } t_{i} \leq t < t_{i+1} \\
      & \frac{(t-t_i)^2(t_{i+2}-t)+(t-t_i)(t_{i+3}-t)(t-t_{i+1})+(t_{i+4}-t)(t-t_{i+1})^2}{6\ \Delta t^3}& \mbox{ if } t_{i+1} \leq t < t_{i+2} \\
      & \frac{(t-t_i)(t_{i+3}-t)^2+(t_{i+4}-t)(t-t_{i+1})(t_{i+3}-t)+(t_{i+4}-t)^2(t-t_{i+2})}{6\ \Delta t^3}& \mbox{ if } t_{i+2} \leq t < t_{i+3} \\
      & \frac{(t_{i+4}-t)^3}{6\ \Delta t^3}& \mbox{ if } t_{i+3} \leq t < t_{i+4}
      \end{eqnarray*}

\section{Evaluation and derivatives}

By denoting $\tau_i=t-t_i$, we can rewrite the above expressions as:
      \begin{eqnarray*}
      b_{i,3}(t)=& \frac{\tau_i^3}{6\ \Delta t^3} & \mbox{ if } t_{i} \leq t < t_{i+1} \\
      & \frac{-3\ \tau_{i+1}^3 + 3\ \Delta t\ \tau_{i+1}^2 + 3\ \Delta t^2\ \tau_{i+1}+\Delta t^3}{6\ \Delta t^3}& \mbox{ if } t_{i+1} \leq t < t_{i+2} \\
      & \frac{3\ \tau_{i+2}^3-6\ \Delta t\ \tau_{i+2}^2 + 4\ \Delta t^3}{6\ \Delta t^3}& \mbox{ if } t_{i+2} \leq t < t_{i+3} \\
      & \frac{-\tau_{i+3}^3 + 3\ \Delta t\ \tau_{i+3}^2 - 3\ \Delta t^2\ \tau_{i+3} + \Delta t^3 }{6\ \Delta t^3}& \mbox{ if } t_{i+3} \leq t < t_{i+4}
      \end{eqnarray*}

To evaluate the value and derivatives of $B$ for a given value $t\in [t_3,t_m]$, we need first to determine which terms of Equation~(\ref{eq:def-spline}) are not equal to 0.

Let us assume that we only know
\begin{itemize}
  \item the interval of definition $[t_3,t_m]$ of the spline,
  \item the number $m$ of control points,
  \item the value $t\in[t_3,t_m]$
\end{itemize}
and we want to evaluate the value of the spline at $t$.

\subsubsection*{Algorithm}

First we compute $\Delta t$:
$$
\Delta t = \frac{t_m - t_3}{m-3}
$$
Then, we determine $i\in\{3,\cdots,m-1\}$ such that $t_i \leq t < t_{i+1}$:
\begin{eqnarray*}
&t_i \leq t < t_{i+1} & \mbox{if and only if} \\
&t_3+(i-3)\Delta t \leq t < t_3+(i-2)\Delta t & \mbox{if and only if}\\
&(i-3) \leq \frac{t-t_3}{\Delta t} < (i-2) & \mbox{if and only if}\\
& i = \lfloor 3+\frac{t-t_3}{\Delta t} \rfloor
\end{eqnarray*}
The only non-zero basis functions in $t$ are thus: $b_{i-3,3}$, $b_{i-2,3}$, $b_{i-1,3}$, $b_{i,3}$ and~(\ref{eq:def-spline}) becomes
\begin{eqnarray*}
  B(t) &=& \sum_{j=i-3}^{i} \textbf{P}_j b_{j,3}(t) \\
&=& \begin{array}{ll}\frac{-\tau_{i}^3 + 3\ \Delta t\ \tau_{i}^2 - 3\ \Delta t^2\ \tau_{i} + \Delta t^3 }{6\ \Delta t^3} & \textbf{P}_{i-3} \\
+ \frac{3\ \tau_{i}^3-6\ \Delta t\ \tau_{i}^2 + 4\ \Delta t^3}{6\ \Delta t^3} & \textbf{P}_{i-2} \\
+ \frac{-3\ \tau_{i}^3 + 3\ \Delta t\ \tau_{i}^2 + 3\ \Delta t^2\ \tau_{i}+\Delta t^3}{6\ \Delta t^3} & \textbf{P}_{i-1}\\
+ \frac{\tau_i^3}{6\ \Delta t^3} & \textbf{P}_{i} \end{array}
\end{eqnarray*}
It follows that:
\begin{eqnarray*}
  \dot{B}(t) &=& \sum_{j=i-3}^{i} \textbf{P}_j \dot{b}_{j,3}(t) \\
&=& \begin{array}{ll}\frac{-3\ \tau_{i}^2 + 6\ \Delta t\ \tau_{i} - 3\ \Delta t^2}{6\ \Delta t^3} & \textbf{P}_{i-3} \\
+ \frac{9\ \tau_{i}^2-12\ \Delta t\ \tau_{i}}{6\ \Delta t^3} & \textbf{P}_{i-2} \\
+ \frac{-9\ \tau_{i}^2 + 6\ \Delta t\ \tau_{i} + 3\ \Delta t^2}{6\ \Delta t^3} & \textbf{P}_{i-1}\\
+ \frac{3\ \tau_i^2}{6\ \Delta t^3} & \textbf{P}_{i} \end{array}
\end{eqnarray*}
and
\begin{eqnarray*}
  \ddot{B}(t) &=& \sum_{j=i-3}^{i} \textbf{P}_j \ddot{b}_{j,3}(t) \\
&=& \frac{-\ \tau_{i} + \ \Delta t}{\Delta t^3} \textbf{P}_{i-3}
+ \frac{3\ \tau_{i}-2\ \Delta t}{\Delta t^3} \textbf{P}_{i-2}
+ \frac{-3\ \tau_{i} + \ \Delta t}{\Delta t^3} \textbf{P}_{i-1}
+ \frac{\tau_i}{\Delta t^3} \textbf{P}_{i}
\end{eqnarray*}
\end{document}
